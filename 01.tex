\chapter{A linguagem da lógica proposicional}


\setcounter{page}{1}    % set page to 1 again to start arabic count
\pagenumbering{arabic}


Capítulo 1 de Souza, \textit{Lógica para Ciência da Computação}~\cite{souza_logica_3}.

\vspace{1cm}

%%%%%%%%%%%%%%%%%%%%%%%%%%%%%%%%%%%%%%%%%%%%%%%%%%%%%%%%%%%%
\begin{easylist}
  & Alfabeto: o alfabeto da Lógica Proposicional é composto por
  && Símbolos de pontuação: $( \; )$
  && Símbolos de verdade: $\TRUE \; \FALSE$
  && Símbolos proposicionais: $A \; B \; C \; P \; Q \; R \; A_1 \; A_2 \; A_3 \; a \; b \; c \; \dots$
  &&& Não se usam as letras $V, v, F, f, T$ e $t$ para não confundir com os valores de verdade.
  && Conectivos proposicionais: $ \NOT \; \OR \; \AND \; \IMP \; \BIC$

\SKIP
  
  & Fórmula: as fórmulas da linguagem da lógica proposicional são construídas a partir dos símbolos do alfabeto conforme as regras a seguir:
  && Todo símbolo de verdade é uma fórmula.
  && Todo símbolo proposicional é uma fórmula.
  && Se $H$ é fórmula, $\NOT H$ é fórmula.
  && Se $H$ e $G$ são fórmulas, $(H \OR G)$, $(H \AND G)$, $(H \IMP G)$ e $(H \BIC G)$ são fórmulas.

\SKIP
  
  & Fórmulas mal formadas: são fórmulas não obtidas da definição anterior.

\SKIP
  
  & Ordem de precedência:
  && $\NOT$      \hspace{2.4cm} Precedência maior.
  && $\IMP \BIC$ \hspace{2cm} $A \IMP B \BIC C$ possui duas interpretações.
  && $\AND$
  && $\OR$       \hspace{2.5cm} Precedência menor.

\SKIP
\SKIP
\SKIP
  
  & Comprimento de uma fórmula:
  && Se $H$ é um símbolo proposicional ou de verdade, $\COMP(H) = 1$.
  && Se $H$ é fórmula, $\COMP(\NOT H) = \COMP(H) + 1$.
  && Se $H$ e $G$ são fórmulas:
  &&& $\COMP(H \OR  G) = \COMP(H) + \COMP(G) + 1$.
  &&& $\COMP(H \AND G) = \COMP(H) + \COMP(G) + 1$.
  &&& $\COMP(H \IMP G) = \COMP(H) + \COMP(G) + 1$.
  &&& $\COMP(H \BIC G) = \COMP(H) + \COMP(G) + 1$.

\SKIP
  
  & Subfórmulas:
  && $H$ é subfórmula de $H$.
  && Se $H = \NOT G$, $G$ é subfórmula de $H$.
  && Se $H$ é uma fórmula do tipo $(G \OR E)$, $(G \AND E)$, $(G \IMP E)$ ou $(G \BIC E)$, então $G$ e $E$ são subfórmulas de $H$.
  && Se $G$ é subfórmula de $H$, então toda subfórmula de $G$ é subfórmula de $H$.

\end{easylist}

%%%%%%%%%%%%%%%%%%%%%%%%%%%%%%%%%%%%%%%%%%%%%%%%%%%%%%%%%%%%
%\section{Componentes de um sistema de processamento de imagens}
