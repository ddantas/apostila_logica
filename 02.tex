\chapter{A semântica da lógica proposicional}


Capítulo 2 de Souza, \textit{Lógica para Ciência da Computação}~\cite{souza_logica_3}.

\vspace{1cm}

%%%%%%%%%%%%%%%%%%%%%%%%%%%%%%%%%%%%%%%%%%%%%%%%%%%%%%%%%%%%
\begin{easylist}
  & Função: é uma relação entre dois conjuntos que associa cada elemento do conjunto de entrada a um único elemento do conjunto de saída

\SKIP
  
  & Função binária: é uma função em que seu contradomínio possui apenas dois elementos

\SKIP
  
  & Interpretação $I$ é uma função binária tal que:
  && O domínio de $I$ é constituído pelo conjunto de fórmulas da lógica proposicional.
  && O contradomínio de $I$ é o conjunto $\{T, F\}$.
  && $I(\TRUE) = T,  \; I(\FALSE) = F$.
  && Se $P$ é um símbolo proposicional, $I(P) \in \{T, F\}$.

\SKIP
  
  & Interpretação de fórmulas: dadas uma fórmula $E$ e uma interpretação $I$, o significado ou interpretação de $E$, denotado por $I(E)$, é determinado pelas regras:
  && Se $E = P$, onde $P$ é um símbolo proposicional, então $I(E) = I(P)$, onde $I(P) \in \{T, F\}$.
  && Se $E = \TRUE$,  então $I(E) = I(\TRUE)  = T$.
  && Se $E = \FALSE$, então $I(E) = I(\FALSE) = F$.
  && Seja $H$ uma fórmula, se $E = \NOT H$ então:
  &&& $I(E) = I(\NOT H) = T \DBIC I(H) = F$.
  &&& $I(E) = I(\NOT H) = F \DBIC I(H) = T$.

\SKIP

  && Sejam $H$ e $G$ duas fórmulas, se $E = (H \OR G)$ então:
  &&& $I(H) = T$ e/ou   $I(G) = T \DBIC I(E) = I(H \OR G) = T$.
  &&& $I(H) = F$ \;e\;  $I(G) = F \DBIC I(E) = I(H \OR G) = F$.
  && Sejam $H$ e $G$ duas fórmulas, se $E = (H \AND G)$ então:
  &&& $I(H) = T$ \;e\;  $I(G) = T \DBIC I(E) = I(H \AND G) = T$.
  &&& $I(H) = F$ e/ou   $I(G) = F \DBIC I(E) = I(H \AND G) = F$.
  && Sejam $H$ e $G$ duas fórmulas, se $E = (H \IMP G)$ então:
  &&& $I(H) = T$ então  $I(G) = T \DBIC I(E) = I(H \IMP G) = T$.
  &&& $I(H) = F$ e/ou   $I(G) = T \DBIC I(E) = I(H \IMP G) = T$.
  &&& $I(H) = T$ \;e\;  $I(G) = F \DBIC I(E) = I(H \IMP G) = F$.
  && Sejam $H$ e $G$ duas fórmulas, se $E = (H \BIC G)$ então:
  &&& $I(H) =    I(G) \DBIC I(E) = I(H \BIC G) = T$.
  &&& $I(H) \neq I(G) \DBIC I(E) = I(H \BIC G) = F$.

\SKIP

\end{easylist}

%%%%%%%%%%%%%%%%%%%%%%%%%%%%%%%%%%%%%%%%%%%%%%%%%%%%%%%%%%%%
%\section{Componentes de um sistema de processamento de imagens}
