\chapter{Propriedades semânticas da lógica proposicional}


Capítulo 3 de Souza, \textit{Lógica para Ciência da Computação}~\cite{souza_logica_3}.

\vspace{1cm}


%%%%%%%%%%%%%%%%%%%%%%%%%%%%%%%%%%%%%%%%%%%%%%%%%%%%%%%%%%%%
\section{Propriedades semânticas}
\label{lprop:propriedadesSemanticas}

\begin{easylist}
  & Tautologia: uma fórmula $H$ é tautologia ou válida se e somente se (sse) para toda interpretação $I$ \[ I(H) = T \]

\SKIP
  
  & Satisfatibilidade: uma fórmula $H$ é satisfatível ou factível se e somente se (sse) existe pelo menos uma interpretação $I$ tal que \[ I(H) = T \]

\SKIP
  
  & Contingência: uma fórmula $H$ é uma contingência se e somente se (sse) existem interpretações $I$ e $J$ tais que \[ I(H) = T \mbox{ e } J(H) = F \]

\SKIP
  
  & Contradição: uma fórmula $H$ é contraditória se e somente se (sse) para toda interpretação $I$ \[ I(H) = F \]

\SKIP
  
  & Implicação: dadas duas fórmulas $H$ e $G$, $H \ISEM G$ ($H$ implica $G$) sse para toda interpretação $I$ \[ \mbox{ se } I(H) = T \mbox{ então } I(G) = T \]

\SKIP
  
  & Equivalência: dadas duas fórmulas $H$ e $G$, $H$ equivale a $G$ sse para toda interpretação $I$ \[ I(H) = I(G) \]

\SKIP
  
  & Dada uma fórmula $H$ e uma interpretação $I$, dizemos que $I$ satisfaz $H$ se \[ I(H) = T \]

\SKIP
  
  & Um conjunto de fórmulas $\beta = \{H_1, H_2, \dots, H_n\}$ é satisfatível sse existe interpretação $I$ tal que \[ I(H_1) = I(H_2) = \dots = I(H_n) = T \]

\SKIP

  & Um conjunto de fórmulas $\beta = \{H_1, H_2, \dots, H_n\}$ é insatisfatível sse não existe interpretação $I$ tal que \[ I(H_1) = I(H_2) = \dots = I(H_n) = T \]

\end{easylist}


%%%%%%%%%%%%%%%%%%%%%%%%%%%%%%%%%%%%%%%%%%%%%%%%%%%%%%%%%%%%
\section{Relações entre propriedades semânticas}

\begin{easylist}
  & Proposição 3.1: seja $H$ uma fórmula, \[H \text{ é tautologia } \DIMP H \text{ é satisfatível. }\]
  
  && Demonstração: $H$ é tautologia $\DBIC$
  para toda interpretação $I$, $I(H) = T \DIMP$
  existe interpretação $I$ tal que $I(H) = T \DBIC$
  $H$ é satisfatível. $\QED$

\SKIP

  & Proposição 3.3: seja $H$ uma fórmula, \[H \text{ é tautologia } \DIMP H \text{ não é contingência. }\]
  
  && Demonstração: $H$ é tautologia $\DBIC$
  para toda interpretação $I$, $I(H) = T \DBIC$
  não existe interpretação $I$ tal que $I(H) = F \DIMP$
  não existem interpretações $I$ e $J$ tais que $I(H) = F$ e $J(H) = T \DBIC$
  $H$ não é contingência. $\QED$

\SKIP

  & Proposição 3.4: seja $H$ uma fórmula, \[H \text{ é contingência } \DIMP H \text{ é satisfatível. }\]
  
  && Demonstração: $H$ é contingência $\DBIC$
  existem interpretações $I$ e $J$ tais que $I(H) = T$ e $J(H) = F$ $\DIMP$
  existe interpretação $I$ tal que $I(H) = T$ $\DBIC$
  $H$ é satisfatível. $\QED$

\SKIP

  & Proposição 3.5: seja $H$ uma fórmula, \[H \text{ é tautologia } \DBIC \; \NOT H \text{ é contraditória. }\]
  
  && Demonstração: $H$ é tautologia $\DBIC$
  para toda interpretação $I$, $I(H) = T \DBIC$
  para toda interpretação $I$, $I(\NOT H) = F \DBIC$
  $\NOT H$ é contraditória. $\QED$

\SKIP

  & Proposição 3.7: sejam $H$ e $G$ duas fórmulas, \[H \text{ equivale a } G \DBIC \; (H \BIC G) \text{ é tautologia. }\]
  
  && Demonstração: $H$ equivale a $G \DBIC$
  para toda interpretação $I$, $I(H) = I(G) \DBIC$
  para toda interpretação $I$, $I(H \BIC G) = T \DBIC$
  $(H \BIC G)$ é tautologia. $\QED$

\SKIP

  & Proposição 3.8: sejam $H$ e $G$ duas fórmulas, \[H \text{ implica } G \DBIC \; (H \IMP G) \text{ é tautologia. }\]
  
  && Demonstração: $H$ implica $G \DBIC$
  para toda interpretação $I$, se $I(H) = T$ então $I(G) = T \DBIC$
  para toda interpretação $I$, $I(H \IMP G) = T \DBIC$
  $(H \IMP G)$ é tautologia. $\QED$

% Teorema da dedução semântica.


\end{easylist}


%%%%%%%%%%%%%%%%%%%%%%%%%%%%%%%%%%%%%%%%%%%%%%%%%%%%%%%%%%%%
\section{Relações semânticas entre os conectivos da lógica proposicional}

\begin{easylist}
  & Conjunto de conectivos completo: o conjunto de conectivos $\psi$ é dito completo se é possível expressar os conectivos $\{\NOT, \OR, \AND, \IMP, \BIC\}$ usando apenas os conectivos de $\psi$.

  && O conectivo $\IMP$ pode ser expresso com $\{\NOT, \OR\}$: \[(P \IMP Q) \equiv \; (\NOT P \OR Q)\]
  && O conectivo $\AND$ pode ser expresso com $\{\NOT, \OR\}$: \[(P \AND Q) \equiv \; \NOT(\NOT P \OR \NOT Q)\]
  && O conectivo $\BIC$ pode ser expresso com $\{\NOT, \OR\}$: \[(P \BIC Q) \equiv \; \NOT(\NOT(\NOT P \OR Q) \OR \NOT(\NOT Q \OR P))\]

\SKIP

  & O conjunto $\{\NOT, \OR\}$ é completo, pois é possível expressar os conectivos $\{\NOT, \OR, \AND, \IMP, \BIC\}$ usando apenas $\{\NOT, \OR\}$.

  & Proposição 3.15 (regra da substituição): sejam $G, G', H$ e $H'$ fórmulas da lógica proposicional tais que:

  && $G$ e $H$ são subfórmulas de $G'$ e $H'$ respectivamente.
  && $G'$ é obtida de $H'$ da substituição de $H$ por $G$ em $H'$.

\[ G \equiv H \DIMP G' \equiv H' \]

\SKIP

  & Definição: o conectivo NAND $(\NAND)$ é definido por $(P \NAND Q) \equiv \; \NOT(P \AND Q)$.

  && O conectivo $\NOT$ pode ser expresso com $\{\NAND\}$: \[(  \NOT P) \equiv \; (P \NAND P)\]
  && O conectivo $\OR$  pode ser expresso com $\{\NAND\}$: \[(P \OR  Q) \equiv \; ((P \NAND P) \NAND (Q \NAND Q))\]

\end{easylist}


%%%%%%%%%%%%%%%%%%%%%%%%%%%%%%%%%%%%%%%%%%%%%%%%%%%%%%%%%%%%
\section{Formas normais na lógica proposicional}

\begin{easylist}

  & Literais: um literal na lógica proposicional é um símbolo proposicional ou sua negação.

  & Forma normal: dada uma fórmula $H$ da lógica proposicional, existe uma fórmula $G$, equivalente a $H$, que está na forma normal. Forma normal é uma estrutura de fórmula pré-definida.

  && Forma normal disjuntiva (FND): é uma disjunção ($\OR$)  de conjunções ($\AND$).
  && Forma normal conjuntiva (FNC): é uma conjunção ($\AND$) de disjunções ($\OR$).

  & Obtenção de formas normais:

  && FND:
  &&& Obtenha a tabela verdade da fórmula.
  &&& Selecione as linhas cuja interpretação é $T$.
  &&& Para cada linha selecionada, faça a conjunção ($\AND$) de todos os símbolos proposicionais cuja interpretação é $T$ com a negação dos símbolos proposicionais cuja interpretação é $F$.
  &&& Faça a disjunção ($\OR$) das fórmulas obtidas no passo anterior.

  && FNC:
  &&& Obtenha a tabela verdade da fórmula.
  &&& Selecione as linhas cuja interpretação é $F$.
  &&& Para cada linha selecionada, faça a disjunção ($\OR$) de todos os símbolos proposicionais cuja interpretação é $F$ com a negação dos símbolos proposicionais cuja interpretação é $T$.
  &&& Faça a conjunção ($\AND$) das fórmulas obtidas no passo anterior.

  \clearpage
  
  & Exemplo: encontre a FND e a FNC da fórmula $((P \IMP Q) \AND R)$.
\end{easylist}

\begin{center}
  \begin{tabular}{ c|c|c|c|c|c|c }
    $P$ & $Q$ & $R$ & $P \IMP Q$ & $(P \IMP Q) \AND R$ & FND & FNC \\
    \hline
    $T$ & $T$ & $T$ & $T$        & $T$                 & $P \AND Q \AND R$ & $ $ \\
    $T$ & $T$ & $F$ & $T$        & $F$                 & $ $ & $\NOT P \OR \NOT Q \OR      R$ \\
    $T$ & $F$ & $T$ & $F$        & $F$                 & $ $ & $\NOT P \OR      Q \OR \NOT R$ \\
    $T$ & $F$ & $F$ & $F$        & $F$                 & $ $ & $\NOT P \OR      Q \OR      R$ \\
    $F$ & $T$ & $T$ & $T$        & $T$                 & $\NOT P \AND Q \AND R$ & $ $ \\
    $F$ & $T$ & $F$ & $T$        & $F$                 & $ $ & $     P \OR \NOT Q \OR      R$ \\
    $F$ & $F$ & $T$ & $T$        & $T$                 & $\NOT P \AND \NOT Q \AND R$ & $ $ \\
    $F$ & $F$ & $F$ & $T$        & $F$                 & $ $ & $     P \OR      Q \OR      R$ \\
  \end{tabular}
\end{center}

\begin{easylist}

  && FND: $(P \AND Q \AND R) \OR (\NOT P \AND Q \AND R) \OR (\NOT P \AND \NOT Q \AND R)$
  && FNC:
    $(\NOT P \OR \NOT Q \OR      R) \AND
     (\NOT P \OR      Q \OR \NOT R) \AND
     (\NOT P \OR      Q \OR      R) \AND \\
     (     P \OR \NOT Q \OR      R) \AND
     (     P \OR      Q \OR      R)$

  
\end{easylist}




  
%%%%%%%%%%%%%%%%%%%%%%%%%%%%%%%%%%%%%%%%%%%%%%%%%%%%%%%%%%%%
\section{Exercícios}

%$ \NOT \; \OR \; \AND \; \IMP \; \BIC$

\begin{enumerate}
  \item Determine o comprimento e o conjunto de subfórmulas das fórmulas a seguir.
    \begin{enumerate}
      \item $P \OR P$
      \item $((\NOT \NOT P \OR Q) \BIC (P \IMP Q)) \AND \TRUE$
      \item $P \IMP ((Q \IMP R) \IMP ((P \IMP R) \IMP (P \IMP R)))$
      \item $((P \IMP \NOT P) \BIC \NOT P) \OR Q$
      \item $\NOT (P \IMP \NOT P)$
    \end{enumerate}
  \item Dentre as concatenações de símbolos a seguir, quais são fórmulas bem formadas e quais são fórmulas mal formadas?
    \begin{enumerate}
      \item $(P \IMP \AND \TRUE)$
      \item $(P \AND Q) \IMP ((Q \BIC P) \OR \NOT \NOT R)$
      \item $\NOT \NOT P$
      \item $\OR Q$
      \item $(P \OR Q) \IMP ((Q \BIC R))$
      \item $PQR$
      \item $A \NOT$
    \end{enumerate}

  \item Demonstre as proposições abaixo usando as regras de interpretação de fórmulas.
    \begin{enumerate}
      \item $I(P \AND Q) = T \Leftrightarrow I(\NOT(\NOT P \OR \NOT Q)) = T$
      \item $I(P \AND Q) = F \Leftrightarrow I(\NOT(\NOT P \OR \NOT Q)) = F$
      \item $I(P \AND Q) = T \Leftrightarrow I(\NOT P \OR \NOT Q) = F$
      \item $I(P \IMP Q) = F \Leftrightarrow I(\NOT P \OR Q) = F$
      \item $I(P \IMP Q) = T \Leftrightarrow I(\NOT P \OR Q) = T$
      \item $I(P \IMP Q) = F \Leftrightarrow I(P \AND \NOT Q) = T$
    \end{enumerate}
  \noindent
  Responda as questões 4, 5 e 6 conforme os exemplos abaixo.
    \begin{enumerate}
      \item Se $I(P) = F$, o que se pode concluir a respeito de $I(H)$?\\
        R: Pode-se concluir que $I(H) = T$.
      \item Se $I(P) = T$, o que se pode concluir a respeito de $I(H)$?\\
        R: Nada se pode concluir.
    \end{enumerate}
  \item Seja $H = (P \IMP Q)$ e $I$ uma interpretação.
    \begin{enumerate}
      \item Se $I(H) = T$, o que se pode concluir a respeito de $I(P)$ e $I(Q)$?
      \item Se $I(H) = T$ e $I(P) = T$, o que se pode concluir a respeito de $I(Q)$?
      \item Se $I(Q) = T$, o que se pode concluir a respeito de $I(H)$?
      \item Se $I(H) = T$ e $I(P) = F$, o que se pode concluir a respeito de $I(Q)$?
      \item Se $I(Q) = F$ e $I(P) = T$, o que se pode concluir a respeito de $I(H)$?
    \end{enumerate}
  \item Seja $I$ uma interpretação tal que $I(P \BIC Q) = T$. O que se pode concluir a respeito de
    \begin{enumerate}
      \item $I(\NOT P \AND Q)$
      \item $I(P \OR \NOT Q)$
      \item $I(Q \IMP P)$
      \item $I((P \AND R) \BIC (Q \AND R))$
      \item $I((P \OR  R) \BIC (Q \OR  R))$
    \end{enumerate}
  \item Repita o exercício anterior considerando $I(P \BIC Q) = F$.

  \item Sejam $H$ e $G$ as fórmulas indicadas a seguir. Identifique, justificando sua resposta, os casos em que $H$ implica $G$.
    \begin{enumerate}
      \item $H = (P \AND Q), G = P$
      \item $H = (P \OR  Q), G = P$
      \item $H = (P \OR \NOT Q), G = \FALSE$
      \item $H = \FALSE, G = P$
      \item $H = P, G = \TRUE$
    \end{enumerate}

  \item Demonstre as proposições abaixo ou dê um contra-exemplo.
    \begin{enumerate}
      \item Proposição 3.6: $H$ não é satisfatível $\Leftrightarrow$ $H$ é contraditória.
      \item $H$ é satisfatível $\Leftrightarrow$ $H$ não é contraditória.
      \item $\NOT H$ é tautologia $\Leftrightarrow$ $H$ é contraditória.
      \item $H$ não é tautologia $\Leftrightarrow$ $H$ é contraditória.
    \end{enumerate}
  
  \item Encontre a FND e a FNC das das fórmulas a seguir.
    \begin{enumerate}
      \item $(P \BIC Q) \AND (P \OR R)$
      \item $(P \IMP Q) \AND (P \IMP R)$
    \end{enumerate}


\end{enumerate}



%%%%%%%%%%%%%%%%%%%%%%%%%%%%%%%%%%%%%%%%%%%%%%%%%%%%%%%%%%%%
\section{Exercícios v.2}

%$ \NOT \; \OR \; \AND \; \IMP \; \BIC$

\begin{enumerate}
  \item Determine o comprimento e o conjunto de subfórmulas das fórmulas a seguir.
    \begin{enumerate}
      \item $P \OR Q$
      \item $\NOT ( P \IMP Q ) \BIC ( R \AND S)$
      \item $ (\NOT P \OR \NOT Q) \BIC \NOT ( P \AND Q )$
      \item $ ( A \AND ( B \AND ( C \AND D ) ) ) \OR ( \NOT A \AND ( \NOT B \AND ( C \AND D ) ) )$
    \end{enumerate}

  \item Dentre as concatenações de símbolos a seguir, quais são fórmulas bem formadas e quais são fórmulas mal formadas?
    \begin{enumerate}
      \item $ P \IMP ( ( Q \IMP R ) \IMP ( ( P \IMP R ) \IMP ( P \IMP R ) ) )$
      \item $ (P \OR \IMP Q) \AND R$
      \item $ ( ( P \IMP \NOT P ) \BIC \NOT P ) \OR Q$
      \item $ P \NOT \IMP Q$
    \end{enumerate}

  \item Demonstre as proposições abaixo usando as regras de interpretação de fórmulas.
    \begin{enumerate}
      \item $I(P \IMP Q) = T \Leftrightarrow I(\NOT ( P \AND Q)) = T$
      \item $I(\NOT P \IMP \NOT Q) = T \Leftrightarrow I(\NOT (\NOT P \AND Q) = T$
    \end{enumerate}
  \noindent
  Responda as questões 4 conforme os exemplos abaixo.
    \begin{enumerate}
      \item Se $I(P) = F$, o que se pode concluir a respeito de $I(H)$?\\
        R: Pode-se concluir que $I(H) = T$.
      \item Se $I(P) = T$, o que se pode concluir a respeito de $I(H)$?\\
        R: Nada se pode concluir.
    \end{enumerate}
  \item O que se pode concluir a respeito de
    \begin{enumerate}
      \item $I(P \AND Q)$ se $I(P) = T$
      \item $I(P \AND Q)$ se $I(P) = F$
      \item $I(P \OR Q)$  se $I(Q) = T$
      \item $I(P \OR Q)$  se $I(Q) = F$
      \item $I(P \IMP Q)$ se $I(P) = T$
      \item $I(P \IMP Q)$ se $I(Q) = T$
      \item $I(P \IMP Q)$ se $I(P) = F$
      \item $I(P \IMP Q)$ se $I(Q) = F$
      \item $I(P \BIC Q)$ se $I(P) = T$
      \item $I(P \BIC Q)$ se $I(P) = F$
    \end{enumerate}

  \item Mostre se os conjuntos de fórmulas a seguir são satisfatíveis ou insatisfatíveis.
    \begin{enumerate}
      \item $\{ (P \AND Q), (P \OR Q) \}$
      \item $\{ (P \AND Q), (P \IMP Q) \}$
      \item $\{ (P \OR Q),  (P \BIC Q) \}$
      \item $\{ (P \AND Q), (P \IMP \NOT Q) \}$
    \end{enumerate}

  \item Demonstre as proposições abaixo ou dê um contra-exemplo.
    \begin{enumerate}
      \item $H$ é satisfatível $\Leftrightarrow \NOT H$ é satisfatível
      \item $H$ é contraditória $\Leftrightarrow \NOT H$ é tautologia
      \item $H$ é tautologia $\Leftrightarrow \NOT H$ é contraditória
      \item $H$ é tautologia $\Rightarrow H$ é satisfatível
      \item $H$ implica $G \Leftrightarrow ( H \IMP G )$ é tautologia
      \item $H$ equivale a $G \Leftrightarrow ( H \BIC G )$ é tautologia
    \end{enumerate}

  \item Demonstre se as fórmulas a seguir são tautologias usando o método da tabela verdade e o da árvore semântica.
    \begin{enumerate}
      \item $H = ( P \OR Q ) \BIC (\NOT P \IMP Q )$
      \item $H = \NOT ( P \BIC Q ) \BIC (\NOT P \BIC Q )$
      \item $H = (\NOT P \BIC \NOT Q ) \BIC \NOT ( P \BIC \NOT Q )$
      \item $H = ( P \OR \NOT Q ) \BIC ( \NOT P  \IMP \NOT Q )$
    \end{enumerate}

  \item Demonstre por absurdo se as fórmulas a seguir são ou não tautologias.
    \begin{enumerate}
      \item $( P \AND Q ) \BIC (\NOT P \OR Q )$
      \item $( P \OR Q ) \BIC (\NOT P \OR Q )$
      \item $( P \AND Q ) \BIC ( P \AND \NOT P )$
    \end{enumerate}

\end{enumerate}

  
  

  
